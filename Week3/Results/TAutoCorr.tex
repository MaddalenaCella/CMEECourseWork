\documentclass[12pt,letterpaper] {article}  

\usepackage{amssymb}

\usepackage{amsmath}

\title {Autocorrelation in weather}

\author {Maddalena Cella}

\date {\today}

\begin{document}
  \maketitle

  \section{Research Question}
  The R script TAutoCorr.R aims to answer the question: \textit{Are temperatures of one year significantly 
  correlated with the next year, across years in a given location?} 

  \section{Results}
  The approximate p-value was calculated as the fraction of the correlation coefficients obtained from the
  permutations (a) that were greater than the correlation coefficient from the unpermuted dataset (b).

  \begin{equation*}
    pvalue = \frac{instances\ of\ a > b}{total\ number\ of\ a(s)}\ = 0.0\\
  \end{equation*}

  \section{Interpretation}
  At a given location, the temperatures of successive years are significantly correlated.

\end{document}
